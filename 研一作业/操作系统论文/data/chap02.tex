% !Mode:: "TeX:UTF-8"
\chapter{结构}

\section{FAT32卷的数据结构排序}
The following table shows the order of the
data structures that compose a FAT32 disk volume.

%\makeatletter\def\@captype{table}\makeatother
%\centering
%\begin{tabular}{ccccc}
%\hline
%Boot & Reserved & FAT & FAT & File\& \\
%Sector & Sectors & (Copy 1) & (Copy 2) & Directory \\
% & & & & Sectors \\
%\hline
%\end{tabular}
%\label{tab:test}

%\begin{table}[htbp]
%	\centering
%	\caption{\label{comparison}Result comparison on LN data}
%	\begin{tabular}{c|c|c|c|c|c|c|c}
%		\hline
%		\multirow{2}{*}{Instance} & \multirow{2}{*}{Original Instance} & \multirow{2}{*}{High Priority} & \multirow{2}{*}{Low Priority} & \multicolumn{2}{|c|}{Benchmark} & \multicolumn{2}{|c}{Our Algorithm} \\
%		\cline{5-8}
%		& & & & Utilization & Time(s) & Utilization & Time(s)\\
%		\hline
%		LN01\&02  &  LN01 \& LN02    &     LN01      &    LN02     &  99.3\%   & 624 &    &   \\
%		\hline
%	\end{tabular}
%\end{table}


\begin{table}[htbp]
	\centering
	\caption[排序结构]{模排序结构}
	\begin{tabular}{ccccc}
	\hline
	Boot & Reserved & FAT & FAT & File\& \\
	Sector & Sectors & (Copy 1) & (Copy 2) & Directory \\
	& & & & Sectors \\
	\hline
	\end{tabular}
\end{table}

\section{FAT32文件系统由四个不同的部分组成}
\begin{enumerate}
	\item 引导扇区位于卷的开头,即第0扇区。 它包括一个名为BPB(BIOS参数块)的区域,它从偏移量11开始,包含一些基本的文件系统信息。 扇区的其余部分通常包含引导加载程序代码
	\item 保留扇区紧跟引导扇区。 包括引导扇区的卷的保留扇区数由引导扇区的偏移量14处的BPB指示。 通常,保留的扇区包括扇区1的文件系统信息扇区和卷的扇区6的备份引导扇区。
	\item 文件分配表是一个32位宽的条目数组,跨越BPB指示的多个扇区,位于引导扇区的偏移量36处。 FAT32通常有两个FAT数据结构副本,以便冗余检查磁盘介质,而一个是FLASH介质。 引导扇区的BPB偏移40的第7位指示FAT是否镜像。 该区域为文件系统和后缀32赋予FAT名称
	\item 文件和目录扇区构成文件系统的其余部分,直到存储实际文件和目录数据的卷的末尾。 FAT32通常会在第一个集群中占用根目录(一个集群是固定数量的连续扇区,该数量由文件和目录扇区的引导扇区的偏移量13处的BPB指示),并由偏移处的BPB指示 引导扇区44。
\end{enumerate}