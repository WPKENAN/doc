% !Mode:: "TeX:UTF-8"
\chapter{数据挖掘相关概念}
\subsection{数据挖掘的基本概念和原理}

对于数据挖掘可以从两个方面进行看待和定义。首先是技术方面的定义,数据挖掘是通过对大量数据进行分析从而发现并提取隐含在其中的具有价值的信息的过程。其次对于商业角度来说,数据挖掘是一种新型的商业信息处理技术,可以从海量数据中进行数据抽样并进行分析以及模型化处理从而提前有利于商业发展的关键数据。通常数据挖掘的任务主要分为两大类:预测任务,描述任务。
\par
数据集的三个主要特性也是研究的重要方面对数据挖掘技术具有重要影响。其分别为维度,稀疏性和分辨率。
\begin{itemize}
	\item 维度:数据集的维度是数据的集中的对象所具有的属性个数。分析较高维度的数据时有可能会出现维灾难。因此,数据预处理的一个重要目的即为降低数据维度。
	\item 稀疏性:指所对应对象的大部分属性上的值为0。稀疏性有利于节约计算和存储的时间。
	\item 分辨率:即衡量数据的尺度,在不同的分辨率下对相同的数据进行分析,得到的性质可能是不同的。
	\item 数据挖掘是多学科交融所产生的领域,其不仅利用了来自统计学的抽样,估计和假设检验。并且还运用了机器学习,人工智能,以及模式识别的数据搜索算法,建模技术,并且还和分布式计算,可视化处理有密切关联
\end{itemize}

\subsection{数据挖掘常用方法与功能}
数据挖掘主要有四种方法与任务:
\begin{enumerate}
	\item 预测建模:为了说明所处理函数为所预测的目标函数变量而进行的建模。主要分为,分类以及回归。两者都是主要是为找出具有相同特性的一部分数据集。前者主要针对离散型数据,后者则主要针对连续型数据。
	\item 关联分析:用来分析并找出一批数据集中所隐藏的可能所具有的强关联关系。并以特征子集或者是蕴含关系的形式表示出来。
	\item 聚类分析:其主要目的在于发现一组具有相似特性的数据集并将其聚集在一起。聚类可用于对社交网络中用户的行为社团进行分组。
	\item 异常检测:识别具有明显不一致特征的数据集。其基本方法是找出实际结果与预测建模所得出结果中相差较大的数据。
\end{enumerate}


