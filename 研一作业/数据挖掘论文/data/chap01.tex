% !Mode:: "TeX:UTF-8"
\chapter{绪论}
\section{研究背景}
社交网络源自网络社交,其起始为电子邮件,互联网的本质上就是计算机之间相互的联网,早期的e-mail仅解决了远程的邮件传输问题,并流传使用至今,同时它也是网络社交的起点。其进一步的发展出的成果BBS,进一步的实现了人们消息之间进行的互通与交流。BBS把网络社交推进了一步使得即时通信和博客成为了社交工具的升级版。他们提高了信息传输速度和并行处理能力。解决了信息发布单一的特点。伴随着网络社交的演变进一步进化,社交网络随之出现。

\section{研究意义}
如果能够有效地分析,社交网络中用户行为、用户特征、用户之间的信息相互交流,并掌握其行为模式以及交流方式,不仅能够帮助运营商更加全面掌握用户需求,从而更新产品、提高用户体验,还能够帮助卖方更好的了解买方需求,从而采取更加有效的网络传销方式和手段,更加吸引买家的注意力,提高用户使用感。进而有助于推动经济发展。并且,社交网络数据采集与分析能够帮助政府及其有关部门在网络媒体上进行合理的舆论宣传,并且可以对网络安全方面进行监控,进而实现绿色上网。

\section{研究现状}

目前,国内对社交网络的研究主要是从社交网络的分析的方向出发,其数据的抓取方法主要包括,搜索引擎爬虫抓取数据,http响应时间所获取的数据,或者网络流量监测来源数据。例如近些年来国家科技研究部门的最新成果BSNiner,其主要功能是可以通过同时应用多种的计算方式从而获取大量且种类各异的数据的处理结果,并且精确度极高;青大企业的自主研发产品iCNiner,其运行速度和精准度已经达到了国际要求的标准。此外我国政府还加大对数据挖掘等相关方面的研发以及投资力度和人才培养,并在全国多所高等院校内成立相关研究机构。
\par
目前,国外数据挖掘方面的最新研究发展主要有对发现相关知识的方法的进一步研究,例如近年来注重对Bayes(贝叶斯)方法以及Boosting方法方面的研究,改进提高;将KDD与数据库进行紧密结合;利用传统的统计学回归方法在KDD中进行应用。在应用方面主要体现在利用KDD等相关商业软件工具从解决孤立的问题过程转向建立问题解决的整体系统,其主要用户包括保险公司、大型银行和销售业等相关企业。许多计算机公司和研究机构都非常重视对数据挖掘相关的开发应用,譬如IBM和微软都相继成立了相应的研究中心。


