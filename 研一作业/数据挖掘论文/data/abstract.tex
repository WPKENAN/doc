% 中英文摘要和关键字
\begin{cabstract}
近些年来,随着社交网络服务蓬勃发展,互联网世界快速进化,社交网络在全球拥有着大量用户,并且使用人数在日益增加,已然成为影响巨大的信息平台。掌握其中有利的信息例如社交网络中用户的行为特征、所具有以及发布的信息所具有的传播规律,甚至内容的定位等,不仅能帮助企业根据客户所需制定出更好更完善及用户友好型的产品,提供更有力的服务,还可以进行更加有效的网络媒体营销。另外,还可为政府及相关部门在舆论控制方面提供有利理论依据和执行的条件,从而优化社会风气。
\par
所以,我选择了基于社交网络的数据采集以及分析。本文中提出并实现了一种,基于新浪微博程序接口和网络数据流相互结合的方式进行数据的采集文献,利用cookie对新浪微博网页端进行了模拟登陆,实现了数据抓取采集。并且通过监测信息交互时传输的数据包。进行动态抓取页面信息。利用python语言网络爬虫和json等数据库函数,解析HTML XML,获得用户的数据信息。然后通过基于图的聚类的社交网络算法以及PageRank算法,进行用户节点数据的建模。收集用户的偏好以及计算其节点的重要性,对微博中用户与用户之间的社交网络关系进行分析,通过多种不同算法对节点的重要性进行了分析,对比多种算法之间的精准度, 对其间的影响力评价模型进行了实现。
\par
本文中所讨论的社交网络用户数据信息的获取,主要问题是在于解决了微博近些年来日渐难以爬取大量数据的问题并通过数据挖掘对对用户行为进行分析研究,得到了具体的用户影响力公式,并根据实际进行了验证与误差分析。

\end{cabstract}

\ckeywords{社交网络 数据采集 数据处理与分析  数据挖掘 图聚类}

\begin{eabstract}
Abstract in English.
\end{eabstract}

\ekeywords{\TeX, \LaTeX, Template, Thesis}
