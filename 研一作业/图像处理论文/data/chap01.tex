% !Mode:: "TeX:UTF-8"
\chapter{课题研究背景以及意义}
\section{研究背景}
汽车的出现改变了以往出行徒步和以马代步的时代,极大地改变了人们的生活方式,扩大了人们的活动范围,加强了人与
人之间的交流。全世界的汽车拥有量呈爆炸性增长,汽车虽方便了我们的出行,但同时也造成了城市交通压力,应用现代
科技解决汽车不断增长而出现的交通问题已经成为一项重要的研究课题,智能交通系统应孕而出。
\section{研究意义}
\par
智能交通系统(Intelligent Transportation System,简称 ITS)是一种充分利用各种先进的高新技术来实现实吋、准确、
高效的交通管理系统,使交通、畅通、安全,它也是一种交通信息服务系统,使人们出行、方便、快捷。随着智能交通系
统的快速发展,智能交通系统已经融入人们的日常生活,使人们的生活越来越方便。车辆是智能交通系统中的重点研究对
象,每辆车都有自身唯一的车牌号码,车牌号码反映了车辆信息以及关联着车主信息,通过车牌号码可以记录对应车辆的
交通行为,因此,车牌识别技术是智能交通系统中最核心最基础的技术之一,决定着智能交通系统的发展速度和技术水
平。
\par
它能够实时地对城市的车辆进行检测、监控和管理,实现智能交通的实时性和高效性;它不仅可以有效地减少人工操
作的参与,节约成本;还可以在一定程度上杜绝一些交通工作人员的违规、舞弊操作,解决收费流失等问题;它还可以对
城市的过往车流量进行检测、指导相关工作,减少交通拥堵现象。
在这个大力倡导智慧型城市概念的社会,随着互联网技术的提升,网络的发展,智能的车牌识别系统早已经深入人们
的生活中,监测车流量等。
\begin{itemize}
\item电子警察系统:一种抓拍车辆违章违规行为的智能系统,大大降低了交通管理压力。
\item卡口系统:对监控路段的机动车辆进行全天候的图像抓拍,自动识别车牌号码,通过公安专网与卡口系统控制中心的
黑名单数据库进行比对,当发现结果相符合时,系统自动向相关人员发出警报信号。
\item高速公路收费系统:自动化管理,当车辆在高速公路收费入口站时,系统进行车牌识别,保存车牌信息,当车辆在高
速公路收费出口站时,系统再次进行车牌识别,与进入车辆的车牌信息进行比对,只有进站和出站的车牌一致方可让车辆
通行。
\item停车场收费系统:随处可见,收费系统抓拍车辆图片进行车牌识别,保存车辆信息和进入时间,并语音播报空闲车
位,当车辆离幵停车场时,收费系统自动识别出该车的车牌号码和保存车辆离幵的时间,并在数据库中查找该车的进入时
间,计算出该车的停车费周,车主交完费用后,收费系统自动放行。
\item智能公交报站:当公交车进入和离开公交站台时,报站系统对其进行车牌识别,然后与数据库中的车牌进行比对,语
音报读车牌结果和公交线路。
\end{itemize}

\section{国内外发展现状}
车牌识别技术应用广泛,当然,上面所指的应用只是其中的一小部分。随着智能交通的迅猛发展,社会对车牌自动识
别的需求量会越来越高多,技术上也会越来越高。
\par
车牌自动识别系统也叫做LPR(License Plate Recognition)系统,目前国内做的比较成熟的产品有:京汉王科技有
限公司开发的“汉王眼”车牌识别系统,厦门宸天电子科技有限公司研发的 Supplate系列,深圳吉通电子有限公司研发
的“车牌通”车牌识别产品、亚洲视觉科技有限公司研发的 VECON­VIS 自动识别系统等。也有很多高校在研究这个课题。

国外相对的在这个方面开始的比较早,同时他们的车牌种类单一,字符简单,容易定位识别有关,取得不错的成就。
\par
关于车牌识别的研究,虽然国内外学者已经作了大量的工作,但仍然存在一些问题。在车辆还比较新的时候,车牌上
的字迹清晰,较容易识别,随着车龄越来越大,车子经过风吹雨淋,车牌难免受到一定程度的磨损,这样就会造成识别的
难度。比如车牌图像的倾斜、车牌自身的磨损、光线的干扰都会影响到定位的精度。
\par
车牌字符识别是在车牌准确定位的基础上,对车牌上的汉字、字母、数字进行有效确认的过程。目前已有的方法很
多,但其效果与实际的要求相差很远,难以适应现代化交通系统高速度、快节奏的要求。因而对字符识别的进一步研究也
同样具有紧迫性和必要性。




